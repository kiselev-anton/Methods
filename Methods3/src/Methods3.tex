\documentclass{article}
\usepackage{fontspec}
\usepackage{polyglossia}
\setmainlanguage{russian} 
\setotherlanguage{english}
\newfontfamily\russianfont[Script=Cyrillic]{Helvetica}

\title{Расчетно-графическая работа №3}
\author{Киселев Антон, МТ-301}
\date{}
\begin{document}
  \maketitle

    1. Постановка задачи

 	Дана СЛАУ $Ax = b$,  где $A =
	\left( \begin{array}{rrr}
 	1.64 & -0.06 & 0.2 \\
	 -0.1 & 0.86 & 0.28 \\
	-0.16 & -0.34 & -0.5 \end{array} \right),  
	  b = \left( \begin{array}{r} 2.12\\ 2.46\\ -2.34 \end{array} \right)$.

	Найти решение системы методами: a) Гаусса, b) Гаусса с выбо\-ром гла\-вного элемента, c) Якоби, d) Зейделя. \\
	
	2. Метод Гаусса
	
	Произведем разложение $A = LU$, где $L$ — нижнетреугольная матри\-ца, $U$ — верхнетреугольная, причем с единичной диагональю. Разло\-жение корректно выполни\-тся, так как на диагонали имеем ненулевые элементы. Система имеет вид $L(Ux) = b$. Решим системы $Lz = b, Ux = z$, получим искомый вектор $x$.\\
	
	3. Метод Гаусса с выбором главного элемента.
	
	Запишем систему в виде $\left( A|b\right)$. Делаем 3 шага. На каждом шаге выби\-раем из элементов, стоящих в текущих и следующих столбцах и стро\-ках, макси\-мальный. Переставляем столбцы с строки так, что он окаже\-тся на диагонали. Нормируем текущую строку и исключаем ее из следующих. Аналогично исключаем в обратном порядке, получим единичную диаго\-наль и вектор $x$ на месте вектора $b$.\\
	
	4. Метод Якоби.
	
	Преобразуем задачу из вида $Ax = b$ к виду $x = Bx+c$. Пусть $A = L + D + R$, где $D$ — диагональ, а $L$ и $R$ — соответственно матрицы из всех элементов, находящихся от нее слева и справа. Тогда: \\ $b = Ax = (L+R+D)x = (L+R)x + Dx\\ Dx = -(L+R)x + b, x = -D^{-1}(L+R)x+D^{-1}b\\ B = -D^{-1}(L+R), c = D^{-1}b$\\
	Убедимся, что $\Vert B \Vert < 1$, тогда получим глобальную сходимость. Применяем метод в виде $x_{k+1} = Bx_{k} + c$, пока $\Vert x_{k+1} - x_{k} \Vert > \varepsilon$, где $\varepsilon$ — точность.\\
	
	5. Метод Зейделя.
	
	Изменим слегка реализацию предыдущего метода, исполь\-зуя уже по\-счи\-тан\-ные координаты вектора $x_{k+1}$ при нахождении следующих, вместо координат $x_{k}$. Остальное оставим тем же.\\
	
	6. Реализация методов и результаты вычислений.
	
	Заметим, что система имеет точное решение $x = \left( \begin{array}{c} 1\\ 2\\ 3 \end{array} \right)$.
	
	Произведем расчеты, используя реализацию методов на языке про\-грам\-ми\-ро\-вания Scala.\\
	
	a) Метод Гаусса.\\ Результат: $x^* = \left( \begin{array}{c} 1.0\\ 2.0\\ 2.9999999999999996 \end{array} \right)$.\\
	 Невязка решения: $\Vert x - x^* \Vert = 4.440892098500626*10^{-16}$\\
	
	b) Метода Гаусса с выбором главного элемента.\\ Результат: $x^* = \left( \begin{array}{c} 1.0000000000000002\\ 2.0\\ 2.999999999999999 \end{array} \right)$.\\
	 Невязки решения:\\ $\Vert x - x^* \Vert_1 = 1.1102230246251565*10^{-15},\\ \Vert x - x^* \Vert_2 = 9.155133597044475*10^{-16},\\ \Vert x - x^* \Vert_\infty = 8.881784197001252*10^{-16}$	\\
	
	c) Метод Якоби. \\ $x_0 = \left( \begin{array}{c} 0\\ 0\\ 0 \end{array} \right)$ Результат: $x^* = \left( \begin{array}{c} 0.9999975033885853\\ 1.9999929889635784\\ 2.9999960925733347 \end{array} \right)$.\\
	Алгоритм совершил 18 итераций.\\
	Невязки решения:\\ $\Vert x - x^* \Vert_1 = 1.3415074501699209*10^{-5},\\ \Vert x - x^* \Vert_2 = 8.4056935113481*10^{-6},\\ \Vert x - x^* \Vert_\infty = 7.01103642164469*10^{-6}$	\\

	d) Метод Зейделя.\\ $x_0 = \left( \begin{array}{c} 0\\ 0\\ 0 \end{array} \right)$ Результат: $x^* = \left( \begin{array}{c} 1.0000017498544882\\ 2.0000036391336073\\2.9999969654357104\end{array} \right)$ \\
	Алгоритм совершил 11 итераций.\\
	Невязки решения:\\ $\Vert x - x^* \Vert_1 = 8.423552385172783*10^{-6},\\ \Vert x - x^* \Vert_2 = 5.051125079610882*10^{-6},\\ \Vert x - x^* \Vert_\infty = 3.6391336073471336*10^{-6}$	\\

\end{document}